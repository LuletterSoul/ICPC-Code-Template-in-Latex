\documentclass[10pt,a4paper]{article}
\usepackage{xeCJK}
\usepackage{amsmath, amsthm}
\usepackage{listings,xcolor}
\usepackage{geometry} % 设置页边距
\usepackage{fontspec}
\usepackage{graphicx}
\usepackage[colorlinks]{hyperref}
\usepackage{setspace}
\usepackage{fancyhdr} % 自定义页眉页脚



\linespread{1.2}

\title{Template For ICPC}
\author{Aether}
\definecolor{dkgreen}{rgb}{0,0.6,0}
\definecolor{gray}{rgb}{0.5,0.5,0.5}
\definecolor{mauve}{rgb}{0.58,0,0.82}

\pagestyle{fancy}

\lhead{\CJKfamily{kai} Template By Aether} %以下分别为左中右的页眉和页脚
\chead{}

\rhead{\CJKfamily{kai} 第 \thepage 页}
\lfoot{} 
\cfoot{\thepage}
\rfoot{}
\renewcommand{\headrulewidth}{0.4pt} 
\renewcommand{\footrulewidth}{0.4pt}
%\geometry{left=2.5cm,right=3cm,top=2.5cm,bottom=2.5cm} % 页边距
\geometry{left=3.18cm,right=3.18cm,top=2.54cm,bottom=2.54cm}
\setlength{\columnsep}{30pt}

\makeatletter

\makeatother



\lstset{
    language    = c++,
    numbers     = left,
    numberstyle={                               % 设置行号格式
        \small
        \color{black}
        \fontspec{Consolas}
    },
	commentstyle = \color[RGB]{0,128,0}\bfseries, %代码注释的颜色
	keywordstyle={                              % 设置关键字格式
        \color[RGB]{40,40,255}
        \fontspec{Consolas Bold}
        \bfseries
    },
	stringstyle={                               % 设置字符串格式
        \color[RGB]{128,0,0}
        \fontspec{Consolas}
        \bfseries
    },
	basicstyle={                                % 设置代码格式
        \fontspec{Consolas}
        \small\ttfamily
    },
	emphstyle=\color[RGB]{112,64,160},          % 设置强调字格式
    breaklines=true,                            % 设置自动换行
    tabsize     = 4,
    frame       = single,%主题
    columns     = fullflexible,
    rulesepcolor = \color{red!20!green!20!blue!20}, %设置边框的颜色
    showstringspaces = false, %不显示代码字符串中间的空格标记
	escapeinside={\%*}{*)},
}

\begin{document}
\title{ICPC Templates}
\author {Shanda}
\maketitle
\tableofcontents
\newpage
\section{基础算法}
\subsection{快速排序}
\subsubsection{模板}
\lstinputlisting{基础算法/快速排序/Quicksort.cpp}
\subsubsection{第k个数}
\lstinputlisting{基础算法/快速排序/第k个数.cpp}
\subsection{归并排序}
\subsubsection{模板}
\lstinputlisting{基础算法/归并排序/Mergesort.cpp}
\subsubsection{逆序对数量}
\lstinputlisting{基础算法/归并排序/逆序对数量.cpp}
\subsection{二分}
\subsubsection{数的范围}
\lstinputlisting{基础算法/二分/数的范围.cpp}
\subsubsection{数的三次方根}
\lstinputlisting{基础算法/二分/数的三次方根.cpp}
\subsection{高精度运算}
\subsubsection{加法}
\lstinputlisting{基础算法/高精度运算/加法.cpp}
\subsubsection{减法}
\lstinputlisting{基础算法/高精度运算/减法.cpp}
\subsubsection{乘法}
\lstinputlisting{基础算法/高精度运算/乘法.cpp}
\subsubsection{除法}
\lstinputlisting{基础算法/高精度运算/除法.cpp}
\subsection{前缀与差分}
\subsubsection{前缀和}
\lstinputlisting{基础算法/前缀与差分/前缀和.cpp}
\subsubsection{子矩阵的和}
\lstinputlisting{基础算法/前缀与差分/子矩阵的和.cpp}
\subsubsection{差分}
\lstinputlisting{基础算法/前缀与差分/差分.cpp}
\subsubsection{差分矩阵}
\lstinputlisting{基础算法/前缀与差分/差分矩阵.cpp}
\subsection{双指针算法}
\subsubsection{最长连续不重复子序列}
\lstinputlisting{基础算法/双指针算法/最长连续不重复子序列.cpp}
\subsubsection{数组元素的目标和}
\lstinputlisting{基础算法/双指针算法/数组元素的目标和.cpp}
\subsubsection{判断子序列}
\lstinputlisting{基础算法/双指针算法/判断子序列.cpp}
\subsection{位运算}
\subsubsection{二进制中1的个数}
\lstinputlisting{基础算法/位运算/二进制中1的个数.cpp}
\subsection{离散化}
\subsubsection{区间和}
\begin{spacing}{1.5}
假定有一个无限长的数轴,数轴上每个坐标上的数都是0。
现在,我们首先进行 n 次操作,每次操作将某一位置x上的数加c。
接下来,进行 m 次询问,每个询问包含两个整数l和r,你需要求出在区间[l, r]之间的所有数的和。
\end{spacing}
\lstinputlisting{基础算法/离散化/区间和.cpp}
\subsection{区间合并}
\subsubsection{区间合并}
\lstinputlisting{基础算法/区间合并/区间合并.cpp}
\section{数据结构}
\subsection{单链表}
\lstinputlisting{数据结构/单链表.cpp}
\subsection{双链表}
\lstinputlisting{数据结构/双链表.cpp}
\subsection{模拟栈}
\lstinputlisting{数据结构/模拟栈.cpp}
\subsection{模拟队列}
\lstinputlisting{数据结构/模拟队列.cpp}
\subsection{单调栈}
\lstinputlisting{数据结构/单调栈.cpp}
\subsection{单调队列}
\lstinputlisting{数据结构/单调队列.cpp}
\subsection{kmp}
\lstinputlisting{数据结构/kmp.cpp}
\subsection{Trie树}
\lstinputlisting{数据结构/Trie树.cpp}
\subsection{最大异或树}
\lstinputlisting{数据结构/最大异或树.cpp}
\subsection{并查集}
\subsubsection{合并集合}
\lstinputlisting{数据结构/并查集/合并集合.cpp}
\subsubsection{连通块中点的数量}
\lstinputlisting{数据结构/并查集/连通块中点的数量.cpp}
\subsubsection{食物链}
\lstinputlisting{数据结构/并查集/食物链.cpp}
\subsection{堆}
\subsubsection{堆排序}
\lstinputlisting{数据结构/堆/堆排序.cpp}
\subsubsection{模拟堆}
\lstinputlisting{数据结构/堆/模拟堆.cpp}
\subsection{哈希表}
\subsubsection{模拟散列表}
\lstinputlisting{数据结构/哈希表/模拟散列表.cpp}
\subsubsection{字符串哈希}
\lstinputlisting{数据结构/哈希表/字符串哈希.cpp}
\section{搜索与图论}
\subsection{DFS}
\subsubsection{排列数字}
\lstinputlisting{搜索与图论/DFS/排列数字.cpp}
\subsubsection{n皇后问题}
\lstinputlisting{搜索与图论/DFS/n皇后问题.cpp}
\subsection{BFS}
\subsubsection{走迷宫}
\lstinputlisting{搜索与图论/BFS/走迷宫.cpp}
\subsubsection{八重码}
\lstinputlisting{搜索与图论/BFS/八重码.cpp}
\subsection{树与图的DFS}
\subsubsection{树的重心}
\lstinputlisting{搜索与图论/树与图的DFS/树的重心.cpp}
\subsection{树与图的BFS}
\subsubsection{图中点的层次}
\lstinputlisting{搜索与图论/树与图的BFS/图中点的层次.cpp}
\subsection{拓扑排序}
\subsubsection{有向图的拓扑排序}
\lstinputlisting{搜索与图论/拓扑排序/有向图的拓扑排序.cpp}
\subsection{Dijkstra}
\subsubsection{求最短路1}
\lstinputlisting{搜索与图论/Dijkstra/求最短路1.cpp}
\subsubsection{求最短路2}
\lstinputlisting{搜索与图论/Dijkstra/求最短路2.cpp}
\subsection{Bellman-Ford}
\subsubsection{有边数限制的最短路}
\lstinputlisting{搜索与图论/Bellman-Ford/有边数限制的最短路.cpp}
\subsection{SPFA}
\subsubsection{SPFA求最短路}
\lstinputlisting{搜索与图论/SPFA/SPFA求最短路.cpp}
\subsubsection{SPFA判断负环}
\lstinputlisting{搜索与图论/SPFA/SPFA判断负环.cpp}
\subsection{Floyd}
\subsubsection{Floyd求最短路}
\lstinputlisting{搜索与图论/Floyd/Floyd求最短路.cpp}
\subsection{最小生成树}
\subsubsection{Kruskal}
\lstinputlisting{搜索与图论/最小生成树/Kruskal.cpp}
\subsubsection{Prim}
\lstinputlisting{搜索与图论/最小生成树/Prim.cpp}
\subsection{二分图}
\subsubsection{染色法判定二分图}
\lstinputlisting{搜索与图论/二分图/染色法判定二分图.cpp}
\subsubsection{匈牙利算法求二分最大匹配}
\lstinputlisting{搜索与图论/二分图/匈牙利算法求二分最大匹配.cpp}
\section{数论}
\subsection{质数}
\subsubsection{试除法判定质数}
\lstinputlisting{数论/质数/试除法判定质数.cpp}
\subsubsection{分解质因数}
\lstinputlisting{数论/质数/分解质因数.cpp}
\subsubsection{筛质数}
\lstinputlisting{数论/质数/筛质数.cpp}
\subsection{约数}
\subsubsection{试除法求约数}
\lstinputlisting{数论/约数/试除法求约数.cpp}
\subsubsection{约数个数}
\lstinputlisting{数论/约数/约数个数.cpp}
\subsubsection{约数之和}
\lstinputlisting{数论/约数/约数之和.cpp}
\subsubsection{最大公约数}
\lstinputlisting{数论/约数/最大公约数.cpp}
\subsection{欧拉函数}
\subsubsection{欧拉函数}
\lstinputlisting{数论/欧拉函数/欧拉函数.cpp}
\subsubsection{筛法求欧拉函数}
\lstinputlisting{数论/欧拉函数/筛法求欧拉函数.cpp}
\subsection{快速幂}
\subsubsection{快速幂}
\lstinputlisting{数论/快速幂/快速幂.cpp}
\subsubsection{快速幂求逆元}
\lstinputlisting{数论/快速幂/快速幂求逆元.cpp}
\subsection{扩展欧几里得算法}
\subsubsection{扩展欧几里得算法}
\lstinputlisting{数论/扩展欧几里得算法/扩展欧几里得算法.cpp}
\subsubsection{线性同余方程}
\lstinputlisting{数论/扩展欧几里得算法/线性同余方程.cpp}
\subsection{中国剩余定理}
\subsubsection{表达整数的奇怪方式}
\lstinputlisting{数论/中国剩余定理/表达整数的奇怪方式.cpp}
\subsection{高斯消元}
\subsubsection{高斯消元解线性方程组}
\lstinputlisting{数论/高斯消元/高斯消元解线性方程组.cpp}
\subsubsection{高斯消元解异或线性方程组}
\lstinputlisting{数论/高斯消元/高斯消元解异或线性方程组.cpp}
\subsection{求组合数}
\subsubsection{上三角法}
\lstinputlisting{数论/求组合数/上三角法.cpp}
\subsubsection{小费马定理与逆元法}
\lstinputlisting{数论/求组合数/小费马定理与逆元法.cpp}
\subsubsection{卢卡斯定理}
\lstinputlisting{数论/求组合数/卢卡斯定理.cpp}
\subsubsection{高精度}
\lstinputlisting{数论/求组合数/高精度.cpp}
\subsubsection{卡特兰数}
\lstinputlisting{数论/求组合数/卡特兰数.cpp}
\subsection{容斥原理}
\subsubsection{能被整除的数}
\lstinputlisting{数论/容斥原理/能被整除的数.cpp}
\subsection{博弈论}
\subsubsection{Nim游戏}
\lstinputlisting{数论/博弈论/Nim游戏.cpp}
\subsubsection{台阶-Nim游戏}
\lstinputlisting{数论/博弈论/台阶-Nim游戏.cpp}
\subsubsection{集合-Nim游戏}
\lstinputlisting{数论/博弈论/集合-Nim游戏.cpp}
\subsubsection{拆分-Nim游戏}
\lstinputlisting{数论/博弈论/拆分-Nim游戏.cpp}
\section{动态规划}
\subsection{背包问题}
\subsubsection{01背包问题}
\lstinputlisting{动态规划/背包问题/01背包问题.cpp}
\subsubsection{完全背包问题}
\lstinputlisting{动态规划/背包问题/完全背包问题.cpp}
\subsubsection{多重背包问题}
\lstinputlisting{动态规划/背包问题/多重背包问题.cpp}
\subsubsection{多重背包问题2}
\lstinputlisting{动态规划/背包问题/多重背包问题2.cpp}
\subsubsection{分组背包问题}
\lstinputlisting{动态规划/背包问题/分组背包问题.cpp}
\subsection{线性DP}
\subsubsection{数字三角形}
\lstinputlisting{动态规划/线性DP/数字三角形.cpp}
\subsubsection{最长上升子序列}
\lstinputlisting{动态规划/线性DP/最长上升子序列.cpp}
\subsubsection{最长上升子序列2}
\lstinputlisting{动态规划/线性DP/最长上升子序列2.cpp}
\subsubsection{最长公共子序列}
\lstinputlisting{动态规划/线性DP/最长公共子序列.cpp}
\subsubsection{最短编辑距离}
\lstinputlisting{动态规划/线性DP/最短编辑距离.cpp}
\subsection{区间DP}
\subsubsection{石子合并}
\lstinputlisting{动态规划/区间DP/石子合并.cpp}
\subsection{计数类DP}
\subsubsection{整数划分}
\lstinputlisting{动态规划/计数类DP/整数划分.cpp}
\subsection{数位统计DP}
\subsubsection{计数问题}
\lstinputlisting{动态规划/数位统计DP/计数问题.cpp}
\subsection{状态压缩DP}
\subsubsection{最短Hamilton路径}
\lstinputlisting{动态规划/状态压缩DP/最短Hamilton路径.cpp}
\subsubsection{蒙德里安的梦想}
\lstinputlisting{动态规划/状态压缩DP/蒙德里安的梦想.cpp}
\subsection{树形DP}
\subsubsection{没有上司的舞会}
\lstinputlisting{动态规划/树形DP/没有上司的舞会.cpp}
\subsection{记忆化搜索}
\subsubsection{滑雪}
\lstinputlisting{动态规划/记忆化搜索/滑雪.cpp}
\end{document}